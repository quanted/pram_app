% Generated by Sphinx.
\def\sphinxdocclass{report}
\documentclass[letterpaper,10pt,english]{sphinxmanual}
\usepackage[utf8]{inputenc}
\DeclareUnicodeCharacter{00A0}{\nobreakspace}
\usepackage{cmap}
\usepackage[T1]{fontenc}
\usepackage{babel}
\usepackage{times}
\usepackage[Bjarne]{fncychap}
\usepackage{longtable}
\usepackage{sphinx}
\usepackage{multirow}


\title{ubertool\_eco Documentation}
\date{June 21, 2014}
\release{0.1}
\author{EPA}
\newcommand{\sphinxlogo}{}
\renewcommand{\releasename}{Release}
\makeindex

\makeatletter
\def\PYG@reset{\let\PYG@it=\relax \let\PYG@bf=\relax%
    \let\PYG@ul=\relax \let\PYG@tc=\relax%
    \let\PYG@bc=\relax \let\PYG@ff=\relax}
\def\PYG@tok#1{\csname PYG@tok@#1\endcsname}
\def\PYG@toks#1+{\ifx\relax#1\empty\else%
    \PYG@tok{#1}\expandafter\PYG@toks\fi}
\def\PYG@do#1{\PYG@bc{\PYG@tc{\PYG@ul{%
    \PYG@it{\PYG@bf{\PYG@ff{#1}}}}}}}
\def\PYG#1#2{\PYG@reset\PYG@toks#1+\relax+\PYG@do{#2}}

\expandafter\def\csname PYG@tok@gd\endcsname{\def\PYG@tc##1{\textcolor[rgb]{0.63,0.00,0.00}{##1}}}
\expandafter\def\csname PYG@tok@gu\endcsname{\let\PYG@bf=\textbf\def\PYG@tc##1{\textcolor[rgb]{0.50,0.00,0.50}{##1}}}
\expandafter\def\csname PYG@tok@gt\endcsname{\def\PYG@tc##1{\textcolor[rgb]{0.00,0.27,0.87}{##1}}}
\expandafter\def\csname PYG@tok@gs\endcsname{\let\PYG@bf=\textbf}
\expandafter\def\csname PYG@tok@gr\endcsname{\def\PYG@tc##1{\textcolor[rgb]{1.00,0.00,0.00}{##1}}}
\expandafter\def\csname PYG@tok@cm\endcsname{\let\PYG@it=\textit\def\PYG@tc##1{\textcolor[rgb]{0.25,0.50,0.56}{##1}}}
\expandafter\def\csname PYG@tok@vg\endcsname{\def\PYG@tc##1{\textcolor[rgb]{0.73,0.38,0.84}{##1}}}
\expandafter\def\csname PYG@tok@m\endcsname{\def\PYG@tc##1{\textcolor[rgb]{0.13,0.50,0.31}{##1}}}
\expandafter\def\csname PYG@tok@mh\endcsname{\def\PYG@tc##1{\textcolor[rgb]{0.13,0.50,0.31}{##1}}}
\expandafter\def\csname PYG@tok@cs\endcsname{\def\PYG@tc##1{\textcolor[rgb]{0.25,0.50,0.56}{##1}}\def\PYG@bc##1{\setlength{\fboxsep}{0pt}\colorbox[rgb]{1.00,0.94,0.94}{\strut ##1}}}
\expandafter\def\csname PYG@tok@ge\endcsname{\let\PYG@it=\textit}
\expandafter\def\csname PYG@tok@vc\endcsname{\def\PYG@tc##1{\textcolor[rgb]{0.73,0.38,0.84}{##1}}}
\expandafter\def\csname PYG@tok@il\endcsname{\def\PYG@tc##1{\textcolor[rgb]{0.13,0.50,0.31}{##1}}}
\expandafter\def\csname PYG@tok@go\endcsname{\def\PYG@tc##1{\textcolor[rgb]{0.20,0.20,0.20}{##1}}}
\expandafter\def\csname PYG@tok@cp\endcsname{\def\PYG@tc##1{\textcolor[rgb]{0.00,0.44,0.13}{##1}}}
\expandafter\def\csname PYG@tok@gi\endcsname{\def\PYG@tc##1{\textcolor[rgb]{0.00,0.63,0.00}{##1}}}
\expandafter\def\csname PYG@tok@gh\endcsname{\let\PYG@bf=\textbf\def\PYG@tc##1{\textcolor[rgb]{0.00,0.00,0.50}{##1}}}
\expandafter\def\csname PYG@tok@ni\endcsname{\let\PYG@bf=\textbf\def\PYG@tc##1{\textcolor[rgb]{0.84,0.33,0.22}{##1}}}
\expandafter\def\csname PYG@tok@nl\endcsname{\let\PYG@bf=\textbf\def\PYG@tc##1{\textcolor[rgb]{0.00,0.13,0.44}{##1}}}
\expandafter\def\csname PYG@tok@nn\endcsname{\let\PYG@bf=\textbf\def\PYG@tc##1{\textcolor[rgb]{0.05,0.52,0.71}{##1}}}
\expandafter\def\csname PYG@tok@no\endcsname{\def\PYG@tc##1{\textcolor[rgb]{0.38,0.68,0.84}{##1}}}
\expandafter\def\csname PYG@tok@na\endcsname{\def\PYG@tc##1{\textcolor[rgb]{0.25,0.44,0.63}{##1}}}
\expandafter\def\csname PYG@tok@nb\endcsname{\def\PYG@tc##1{\textcolor[rgb]{0.00,0.44,0.13}{##1}}}
\expandafter\def\csname PYG@tok@nc\endcsname{\let\PYG@bf=\textbf\def\PYG@tc##1{\textcolor[rgb]{0.05,0.52,0.71}{##1}}}
\expandafter\def\csname PYG@tok@nd\endcsname{\let\PYG@bf=\textbf\def\PYG@tc##1{\textcolor[rgb]{0.33,0.33,0.33}{##1}}}
\expandafter\def\csname PYG@tok@ne\endcsname{\def\PYG@tc##1{\textcolor[rgb]{0.00,0.44,0.13}{##1}}}
\expandafter\def\csname PYG@tok@nf\endcsname{\def\PYG@tc##1{\textcolor[rgb]{0.02,0.16,0.49}{##1}}}
\expandafter\def\csname PYG@tok@si\endcsname{\let\PYG@it=\textit\def\PYG@tc##1{\textcolor[rgb]{0.44,0.63,0.82}{##1}}}
\expandafter\def\csname PYG@tok@s2\endcsname{\def\PYG@tc##1{\textcolor[rgb]{0.25,0.44,0.63}{##1}}}
\expandafter\def\csname PYG@tok@vi\endcsname{\def\PYG@tc##1{\textcolor[rgb]{0.73,0.38,0.84}{##1}}}
\expandafter\def\csname PYG@tok@nt\endcsname{\let\PYG@bf=\textbf\def\PYG@tc##1{\textcolor[rgb]{0.02,0.16,0.45}{##1}}}
\expandafter\def\csname PYG@tok@nv\endcsname{\def\PYG@tc##1{\textcolor[rgb]{0.73,0.38,0.84}{##1}}}
\expandafter\def\csname PYG@tok@s1\endcsname{\def\PYG@tc##1{\textcolor[rgb]{0.25,0.44,0.63}{##1}}}
\expandafter\def\csname PYG@tok@gp\endcsname{\let\PYG@bf=\textbf\def\PYG@tc##1{\textcolor[rgb]{0.78,0.36,0.04}{##1}}}
\expandafter\def\csname PYG@tok@sh\endcsname{\def\PYG@tc##1{\textcolor[rgb]{0.25,0.44,0.63}{##1}}}
\expandafter\def\csname PYG@tok@ow\endcsname{\let\PYG@bf=\textbf\def\PYG@tc##1{\textcolor[rgb]{0.00,0.44,0.13}{##1}}}
\expandafter\def\csname PYG@tok@sx\endcsname{\def\PYG@tc##1{\textcolor[rgb]{0.78,0.36,0.04}{##1}}}
\expandafter\def\csname PYG@tok@bp\endcsname{\def\PYG@tc##1{\textcolor[rgb]{0.00,0.44,0.13}{##1}}}
\expandafter\def\csname PYG@tok@c1\endcsname{\let\PYG@it=\textit\def\PYG@tc##1{\textcolor[rgb]{0.25,0.50,0.56}{##1}}}
\expandafter\def\csname PYG@tok@kc\endcsname{\let\PYG@bf=\textbf\def\PYG@tc##1{\textcolor[rgb]{0.00,0.44,0.13}{##1}}}
\expandafter\def\csname PYG@tok@c\endcsname{\let\PYG@it=\textit\def\PYG@tc##1{\textcolor[rgb]{0.25,0.50,0.56}{##1}}}
\expandafter\def\csname PYG@tok@mf\endcsname{\def\PYG@tc##1{\textcolor[rgb]{0.13,0.50,0.31}{##1}}}
\expandafter\def\csname PYG@tok@err\endcsname{\def\PYG@bc##1{\setlength{\fboxsep}{0pt}\fcolorbox[rgb]{1.00,0.00,0.00}{1,1,1}{\strut ##1}}}
\expandafter\def\csname PYG@tok@kd\endcsname{\let\PYG@bf=\textbf\def\PYG@tc##1{\textcolor[rgb]{0.00,0.44,0.13}{##1}}}
\expandafter\def\csname PYG@tok@ss\endcsname{\def\PYG@tc##1{\textcolor[rgb]{0.32,0.47,0.09}{##1}}}
\expandafter\def\csname PYG@tok@sr\endcsname{\def\PYG@tc##1{\textcolor[rgb]{0.14,0.33,0.53}{##1}}}
\expandafter\def\csname PYG@tok@mo\endcsname{\def\PYG@tc##1{\textcolor[rgb]{0.13,0.50,0.31}{##1}}}
\expandafter\def\csname PYG@tok@mi\endcsname{\def\PYG@tc##1{\textcolor[rgb]{0.13,0.50,0.31}{##1}}}
\expandafter\def\csname PYG@tok@kn\endcsname{\let\PYG@bf=\textbf\def\PYG@tc##1{\textcolor[rgb]{0.00,0.44,0.13}{##1}}}
\expandafter\def\csname PYG@tok@o\endcsname{\def\PYG@tc##1{\textcolor[rgb]{0.40,0.40,0.40}{##1}}}
\expandafter\def\csname PYG@tok@kr\endcsname{\let\PYG@bf=\textbf\def\PYG@tc##1{\textcolor[rgb]{0.00,0.44,0.13}{##1}}}
\expandafter\def\csname PYG@tok@s\endcsname{\def\PYG@tc##1{\textcolor[rgb]{0.25,0.44,0.63}{##1}}}
\expandafter\def\csname PYG@tok@kp\endcsname{\def\PYG@tc##1{\textcolor[rgb]{0.00,0.44,0.13}{##1}}}
\expandafter\def\csname PYG@tok@w\endcsname{\def\PYG@tc##1{\textcolor[rgb]{0.73,0.73,0.73}{##1}}}
\expandafter\def\csname PYG@tok@kt\endcsname{\def\PYG@tc##1{\textcolor[rgb]{0.56,0.13,0.00}{##1}}}
\expandafter\def\csname PYG@tok@sc\endcsname{\def\PYG@tc##1{\textcolor[rgb]{0.25,0.44,0.63}{##1}}}
\expandafter\def\csname PYG@tok@sb\endcsname{\def\PYG@tc##1{\textcolor[rgb]{0.25,0.44,0.63}{##1}}}
\expandafter\def\csname PYG@tok@k\endcsname{\let\PYG@bf=\textbf\def\PYG@tc##1{\textcolor[rgb]{0.00,0.44,0.13}{##1}}}
\expandafter\def\csname PYG@tok@se\endcsname{\let\PYG@bf=\textbf\def\PYG@tc##1{\textcolor[rgb]{0.25,0.44,0.63}{##1}}}
\expandafter\def\csname PYG@tok@sd\endcsname{\let\PYG@it=\textit\def\PYG@tc##1{\textcolor[rgb]{0.25,0.44,0.63}{##1}}}

\def\PYGZbs{\char`\\}
\def\PYGZus{\char`\_}
\def\PYGZob{\char`\{}
\def\PYGZcb{\char`\}}
\def\PYGZca{\char`\^}
\def\PYGZam{\char`\&}
\def\PYGZlt{\char`\<}
\def\PYGZgt{\char`\>}
\def\PYGZsh{\char`\#}
\def\PYGZpc{\char`\%}
\def\PYGZdl{\char`\$}
\def\PYGZhy{\char`\-}
\def\PYGZsq{\char`\'}
\def\PYGZdq{\char`\"}
\def\PYGZti{\char`\~}
% for compatibility with earlier versions
\def\PYGZat{@}
\def\PYGZlb{[}
\def\PYGZrb{]}
\makeatother

\begin{document}

\maketitle
\tableofcontents
\phantomsection\label{index::doc}


Contents:


\chapter{Introduction}
\label{intro:introduction}\label{intro:welcome-to-ubertool-eco-s-documentation}\label{intro::doc}
The ubertool is a website, currently available at \href{http://www.ubertool.org}{http://www.ubertool.org}, that houses algorithms, data, and documentation for the risk assessment of chemicals. EPA (2004) provides an overview of OPP’s ecological risk assessment process:
\href{http://www.epa.gov/espp/consultation/ecorisk-overview.pdf}{http://www.epa.gov/espp/consultation/ecorisk-overview.pdf}


\section{Development team}
\label{intro:development-team}
There are three core roles involved in this process, these roles are:
- Product Owner: represents the stakeholders via stories backlog and priorities (Tom Purucker)
- Development Team: delivers product increments at the end of each sprint
-- ORD Athens: Jon Flaishans, Mike Galvin, Tao Hong, Chance Pascale, Tom Purucker, Marcia Snyder, Kurt Wolfe, Nick Pope
-- OPP Potomac Yard: Andrew Kanarek, Shelley Thawley, Trip, Meredith Fry, Nelson Thurman
-- Potentially other contributors within ORD
- Scrum Master: scrum facilitator who removes impediments for delivering sprint goals/deliverables, performs tasking, bug priority, task followup, etc. (Chance Pascale)

Ancillary roles on the scrum team are:
- Stakeholders: (Bill Eckel, Ed Odenkirchen, Dirk Young, Nelson Thurman, Ron Parker, Katrina White, others identified by Bill Eckel)
- Managers: People who control the environment (Kate Sullivan {[}Branch Chief{]}, Roy Sidle {[}Division Director{]}, John Kenneke {[}CSS Matrix Interface{]}, Tina Bohardi {[}CSS{]}, Jim Cowles {[}EFED Director{]}, Andrew Gillespie


\section{Development process}
\label{intro:development-process}
proceed according to the principles of “scrum” development, an iterative and incremental agile software development process for developing applications.
\href{http://en.wikipedia.org/wiki/Scrum\_\%28development\%29}{http://en.wikipedia.org/wiki/Scrum\_\%28development\%29}


\section{Sprint history}
\label{intro:sprint-history}

\chapter{Features}
\label{features::doc}\label{features:features}
ubertool stuff


\section{subsection}
\label{features:subsection}
smaller point


\chapter{Models}
\label{models:models}\label{models::doc}
ubertool stuff


\section{subsection}
\label{models:subsection}
smaller point


\chapter{Architecture}
\label{architecture::doc}\label{architecture:architecture}
ubertool stuff


\section{Technology stack}
\label{architecture:technology-stack}
smaller point


\section{Front-end server}
\label{architecture:front-end-server}

\section{Back-end server}
\label{architecture:back-end-server}

\section{D4EM server}
\label{architecture:d4em-server}

\section{Public deployment}
\label{architecture:public-deployment}

\section{EPA intranet deployment}
\label{architecture:epa-intranet-deployment}

\chapter{Deployment}
\label{deployment::doc}\label{deployment:deployment}
ubertool stuff


\section{subsection}
\label{deployment:subsection}
smaller point


\chapter{Installation}
\label{installation:installation}\label{installation::doc}
ubertool stuff


\section{subsection}
\label{installation:subsection}
smaller point


\chapter{Research}
\label{processes::doc}\label{processes:research}
ubertool stuff


\section{subsection}
\label{processes:subsection}
smaller point


\chapter{API}
\label{api:api}\label{api::doc}
ubertool stuff


\section{subsection}
\label{api:subsection}
smaller point


\chapter{Quality Assurance Project Plan}
\label{qapp:quality-assurance-project-plan}\label{qapp::doc}
ubertool stuff


\section{Problem Statement}
\label{qapp:problem-statement}
Summary of relevant literature

Context of the research - institutional

Problem definition, five elements or ``gap in knowledge''

Projected outcomes and findings

Relationship to other studies; cooperation

Summary of costs and benefits


\section{Statement of Question, Objectives, and Hypotheses}
\label{qapp:statement-of-question-objectives-and-hypotheses}
Overarching research question, derived from problem statement

Intended outcome of this piece of research (usually to test hypotheses and define relationships), including an estimate of time and resources needed

Specific mechanistic hypotheses, tests in summary, populations to which the hypotheses are to be applied

Methods
\begin{enumerate}
\item {} 
List of variables and sources of variation, sorted byindepen- dent and dependent variables, with reasons for their selection

\item {} 
Other sources of variation and how they will be dealt with

\item {} 
Study design and analysis, including models, statistical tests if used, detailed analytical procedures, graphs of potential outcomes

\item {} 
Field, laboratory, and computational procedures, in sufficient detail so that someone other than the author could do the study

\end{enumerate}
\begin{enumerate}
\setcounter{enumi}{6}
\item {} 
Budget and Schedule

\end{enumerate}
\begin{enumerate}
\item {} 
A comprehensive three-column budget for the duration of the study

\item {} 
A schedule of tasks with initiation and completion target dates, with designated responsibilities and reporting requirements and dates

\end{enumerate}
\begin{enumerate}
\setcounter{enumi}{2}
\item {} 
Reports and Publications

\end{enumerate}
\begin{enumerate}
\item {} 
Intended disposition of research results, in terms of audience, publication type, and timing

\item {} 
Fiscal, accounting, and procedural reporting requirements and how they will be met

\end{enumerate}


\chapter{Research}
\label{research::doc}\label{research:research}
ubertool stuff


\section{subsection}
\label{research:subsection}
smaller point


\chapter{Publications}
\label{publications::doc}\label{publications:publications}
ubertool stuff


\section{Peer-reviewed articles}
\label{publications:peer-reviewed-articles}
smaller point


\section{In prep articles}
\label{publications:in-prep-articles}

\section{Other manuscripts}
\label{publications:other-manuscripts}

\section{Abstracts}
\label{publications:abstracts}

\chapter{References}
\label{references:references}\label{references::doc}
ubertool stuff


\section{subsection}
\label{references:subsection}
smaller point


\chapter{Indices and tables}
\label{index:indices-and-tables}\begin{itemize}
\item {} 
\emph{genindex}

\item {} 
\emph{modindex}

\item {} 
\emph{search}

\end{itemize}



\renewcommand{\indexname}{Index}
\printindex
\end{document}
